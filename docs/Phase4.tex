% THIS IS SIGPROC-SP.TEX - VERSION 3.1
% WORKS WITH V3.2SP OF ACM_PROC_ARTICLE-SP.CLS
% APRIL 2009
%
% It is an example file showing how to use the 'acm_proc_article-sp.cls' V3.2SP
% LaTeX2e document class file for Conference Proceedings submissions.
% -------------------------------------------------------------------------------------------------
% This .tex file (and associated .cls V3.2SP) *DOES NOT* produce:
%       1) The Permission Statement
%       2) The Conference (location) Info information
%       3) The Copyright Line with ACM data
%       4) Page numbering
% -------------------------------------------------------------------------------------------------
% It is an example which *does* use the .bib file (from which the .bbl file
% is produced).
% REMEMBER HOWEVER: After having produced the .bbl file,
% and prior to final submission,
% you need to 'insert'  your .bbl file into your source .tex file so as to provide
% ONE 'self-contained' source file.
%
% Questions regarding SIGS should be sent to
% Adrienne Griscti ---> griscti@acm.org
%
% Questions/suggestions regarding the guidelines, .tex and .cls files, etc. to
% Gerald Murray ---> murray@hq.acm.org
%
% For tracking purposes - this is V3.1SP - APRIL 2009
%
%
% NOTE: this document includes multiple images, located in the shared drive

\documentclass[sigconf]{acmart}
\def\BibTeX{\textsc{Bib}\TeX}
\usepackage{url}
\usepackage{balance}

\begin{document}

\title{Secure Chat Application}
\subtitle{Encrypted Communication}

\author{Samuel Lewis}
\affiliation{%
 \institution{Rochester Institute of Technology}
 %   \streetaddress{P.O. Box 1212}
 \city{Rochester}
 \state{New York}
 \postcode{14623}
}
\email{srl8336@rit.edu}

\author{Jonathan Lo}
\affiliation{%
 \institution{Rochester Institute of Technology}
 %   \streetaddress{P.O. Box 1212}
 \city{Rochester}
 \state{New York}
 \postcode{14623}
}
\email{jcl5201@rit.edu}

\author{Jason Tu}
\affiliation{%
 \institution{Rochester Institute of Technology}
 %   \streetaddress{P.O. Box 1212}
 \city{Rochester}
 \state{New York}
 \postcode{14623}
}
\email{jwt8264@rit.edu}

\date{15 February 2017}

\begin{abstract}
 Our project, named \textit{Lil' (Secure) Bits}, will implement a secure chat application that
 integrates existing Computer Security and Network Security principles.
\end{abstract}

\terms{Computer Security, Network Communication, Encryption}

\keywords{Asymmetric Encryption, Ciphertext, Digital Signature, Plaintext, Privacy, Public-Key
Cryptography, Symmetric Encryption} % NOT required for Proceedings

\maketitle


%--------------------------------------------------------------------------------------------------
% Introduction
%--------------------------------------------------------------------------------------------------
\section{Introduction}

\subsection{Motivation}
Our team has a large interest in network-related problems. Given the opportunity to incorporate
this into a security-related problem, and an increased awareness of privacy issues, we decided to
work on an encrypted chat application, to secure our private communication. The team believes that
privacy is a priority; thus, it is necessary to encrypt our messages to prevent any breaches by
unauthorized individuals or organizations. Another important motivation for this project was the
desire to produce a product that could feasibly be used in every day life.

\subsection{Project Goals}

\subsubsection{Purpose}
Project: \texttt{Lil' (Secure) Bits} is designed to provide a secure communication channel that
integrates existing Computer Security principles and Network Security principles. The project will
be used to explore the listed principles and implement a software application that will utilize
encryption algorithms to ensure message communications are secured over the network.

The application goal is the creation of a secure and private communication
channel~\cite{Huang:privacy}. Thus, the team attempts to explore possible
principles and algorithms that will ensure that the communication has satisfied our goals.

\subsubsection{Scope}
The project will focus on utilizing computer security principles, such as existing encryption
algorithms, digital signatures, Advanced Encryption Standard and privacy, to prototype a secure
chat application in an academic environment. The project shall be limited to the listed design
constraints and project goals.


%--------------------------------------------------------------------------------------------------
% Overall Approach
%--------------------------------------------------------------------------------------------------
\section{Overall Approach}

\subsection{Overview}
The application shall be implemented using the client-server model; thus dividing the application
into two network components:
\begin{enumerate}
 \item client component
 \item server component
\end{enumerate}

The server component shall be defined as the "back-end service", which shall provide all the
necessary logistics, such as routing, session management, etc. Following our project's goals, the
design and implementation of the server component shall be restricted. The server component shall
have limited, if any, access to the communication channel/data.

The client component shall be defined as the "front-end service", which shall provide all the
necessary interface for users and shall also include security features, such as encryption, digital
signatures, etc. Following our project's goals, the design and implementation of the client
component shall provide users confidentiality, integrity, and availability. Communications between
the client and server components shall be secured using necessary security features, such as
digital signatures ~\cite{Fersch:provable} ~\cite{Meijer:signature}, public key encryption
~\cite{Kuchlin:publickey}, or Advance Encryption Standard ~\cite{Jariwala:taxonomy}

\subsection{Constraints}

\subsubsection{Design Constraints}
\begin{itemize}
 \item The application must be implemented using Java.
 \item The application must secure communication using existing security principles.
 \item The application must include a user interface.
 \item The application must ensure the following security principles:
       \begin{itemize}
        \item User Identity
        \item Data Confidentiality
        \item Data Integrity
        \item Availability
        \item Access Control
        \item Recovery
       \end{itemize}
\end{itemize}

\subsubsection{External Constraints}
\begin{itemize}
 \item The performance of the system and application is limited to the provided system hardware.
       Therefore, the application features should be limited to the system specifications.
 \item The application shall be implemented for a privately-owned system. Therefore, there are
       limitations on the security on these systems. The team will be implementing the
       application on the assumption that the system itself, as a whole, has been secured.
\end{itemize}


%--------------------------------------------------------------------------------------------------
% Design
%--------------------------------------------------------------------------------------------------
\section{Design}

\subsection{Modules}
Our team decided to divide project into four logical modules, listed below
\begin{itemize}
 \item \texttt{Encryption}
 \item \texttt{Manager}
 \item \texttt{Network}
 \item \texttt{User Interface}
\end{itemize}

\subsubsection{User Interface}
The \texttt{User Interface} serves as a view for the user, allowing easy access to the functions of the chat application. It will be implemented as an user interface, where the user may type commands and feed input, which will be interpreted and sent to the \texttt{Manager} for processing. The \texttt{Manager} will respond back with output, which will be displayed back to the user over the command line.

\subsubsection{Manager}
The \texttt{Manager} module serves as a central hub, connecting and abstracting the functionality
of each module to the others. The \texttt{User Interface} module will take user input and,
communicate the intention and data to the \texttt{Manager}. The \texttt{Manager} will then carry
out the request of the user, whether it be connecting to another user, or sending a message. When
necessary, the \texttt{Manager} will use the \texttt{Encryption} module to encrypt data before
sending, or decrypting received data. The \texttt{Manager} will also use the \texttt{Network}
module to send data out over the network to other users.

\subsubsection{Encryption}
The \texttt{Encryption} module provides the necessary encryption framework for encrypting a
character stream. Keys are safely stored and generated inside, with no functions or public
variables that could leak secure information. Simple encryption related functions are provided to
make using the Encryption object a simple black box experience.

\subsubsection{Network}
The \texttt{Network} module provides the necessary network framework for sending and receiving
data through an established connection. This module establishes the client and server
connection and provides simple network related functions to access the connections.

\subsection{Software Design}

\subsubsection{Modules Architecture}
Figure~\ref{Modules Architecture} outlines the module interfaces.
\begin{figure}[htb]
 \begin{center}
  \includegraphics[width=3in]{Design_ModuleInterfaces.png}
  \caption{Modules Architecture}
  \label{Modules Architecture}
 \end{center}
\end{figure}


%--------------------------------------------------------------------------------------------------
% Scheduling and Planning
%--------------------------------------------------------------------------------------------------
\section{Planning, Requirements, and Deliverables}

\subsection{Phase 2}
Outlines our teams scheduling and planning for Phase 2.

\subsubsection{Planning and Requirements}
Table~\ref{Phase 2 Goals} outlines the team's goals for Phase 2.
\begin{table}[htb]
 \centering
 \caption{Phase 2 Goals}
 \label{Phase 2 Goals}

 \begin{tabular}{|c|} \hline
  \textbf{Goals}                                \\ \hline
  Completion of design and architecture         \\ \hline
  Initial implementation of \texttt{Encryption} \\ \hline
  Initial implementation of \texttt{Network}    \\ \hline
  Initial implementation of \texttt{Manager}    \\ \hline
 \end{tabular}
\end{table}

\subsubsection{Deliverables and Submission}
Table~\ref{Phase 2 Deliverables} outlines the team's deliverables and submission details for Phase
2.
\begin{table}[htb]
 \centering
 \caption{Phase 2 Deliverables}
 \label{Phase 2 Deliverables}

 \begin{tabular}{|c|} \hline
  \textbf{Deliverables}                         \\ \hline
  Completion of design and architecture         \\ \hline
  Initial implementation of \texttt{Encryption} \\ \hline
  Initial implementation of \texttt{Network}    \\ \hline
  Initial implementation of \texttt{Manager}    \\ \hline
 \end{tabular}
\end{table}

\subsection{Phase 3}
Outlines our teams scheduling and planning for Phase 3.

\subsubsection{Planning and Requirements}
Table~\ref{Phase 3 Goals} outlines the team's goals for Phase 3.
\begin{table}[htb]
 \centering
 \caption{Phase 3 Goals}
 \label{Phase 3 Goals}

 \begin{tabular}{|c|} \hline
  \textbf{Goals}                                       \\ \hline
  Implementation of \texttt{Encryption}                \\ \hline
  Implementation of \texttt{Network}                   \\ \hline
  Implementation of \texttt{Manager}                   \\ \hline
  Implementation of \texttt{User Interface}            \\ \hline
  Implementation of thread-safe parallel communication \\ \hline
  Implementation of improved exception handling        \\ \hline
  Implementation of correct end-state termination      \\ \hline
 \end{tabular}
\end{table}
\newline
The full implementation of \texttt{Encryption}, \texttt{Network}, \texttt{Manager} modules includes
the following requirements:
\begin{enumerate}
 \item validate data confidentiality, integrity and availability. As this is an important
       part of our project goal, the team must test the application to verify this
       requirement.
 \item allow a peer-to-peer connection with two clients, one of which acting as the server.
       The application will not support a group messaging system.
 \item provide complete documentation of the implementations of the modules. The team shall
       provide metrics and documentations to report necessary information.
\end{enumerate}

The implementation of \texttt{User Interface} module includes the following requirements:
\begin{enumerate}
 \item allow users to provide input to the module.
 \item provides users with accessible output from the module.
 \item provide complete documentation of the implementation of the module. The team shall provide
       metrics and documentations to report necessary information.
\end{enumerate}
The \texttt{User Interface} module is designed for the purpose of a interactive component to the
application, which is further used to test the application. The amount of effort to implement this
module shall reflect this.

The implementation of thread safe, parallel communication includes the following requirements:
\begin{enumerate}
 \item allow clients to communicate with one another in parallel. This allows the application to
       properly mimic a chat application.
 \item utilize threading mechanics to implement parallel communication. Due to the parallel
       nature, the implementation shall incorporate threading of communication, and other
       input/output operations. We do not require the Network to be threaded as TCP protocol
       itself handles this.
\end{enumerate}

The implementation of improved exception handling includes the following
requirements:
\begin{enumerate}
 \item the application shall gracefully terminate when an unrecoverable fault occurs. The
       default solution to solve any unrecoverable fault is to reset the application, which
       involves terminating.
 \item exceptions caught shall be outputted for debugging. It is important for debugging purposes
       to output any warnings and errors detected, allowing the team to resolve the issue.
\end{enumerate}

The implementation of correct end-state termination includes the following requirements:
\begin{enumerate}
 \item upon termination, allocated resources should be correctly terminated. This includes all
       streams, buffers, and connections.
\end{enumerate}

\subsubsection{Deliverables and Submissions}
Table~\ref{Phase 3 Deliverables} outlines the team's deliverables and submission details for Phase
3.
\begin{table}[htb]
 \centering
 \caption{Phase 3 Deliverables}
 \label{Phase 3 Deliverables}

 \begin{tabular}{|c|} \hline
  \textbf{Deliverables}                                \\ \hline
  Implementation of \texttt{Encryption}                \\ \hline
  Implementation of \texttt{Network}                   \\ \hline
  Implementation of \texttt{Manager}                   \\ \hline
  Implementation of \texttt{User Interface}            \\ \hline
  Implementation of thread-safe parallel communication \\ \hline
  Implementation of improved exception handling        \\ \hline
  Implementation of correct end-state termination      \\ \hline
 \end{tabular}
\end{table}


\subsection{Phase 4}
Outlines our teams scheduling and planning for Phase 4.

\subsubsection{Investigation Criteria}
The following describes the high level criteria for Phase 4 investigations.
Projects will be reviewed under the following criteria in descending priority:
\begin{enumerate}
 \item implementation
 \item execution and testing
 \item result analysis
 \item documentation
\end{enumerate}

\subsubsection{Investigation Plan}
The following describes the high level plan for Phase 4 investigations. Projects
will be investigated under the following plan:
\begin{enumerate}
 \item code review analysis
 \item execution and testing
 \item retrospective
\end{enumerate}

\subsubsection{Investigation Reports}
Investigation reports are supplied specific to each project. Reports will be
included in the package.


%--------------------------------------------------------------------------------------------------
% Implementation
%--------------------------------------------------------------------------------------------------
\section{Implementation}

\subsection{Manager Module}
\subsubsection{Manager Class}
Figure~\ref{Manager UML} outlines the Manager class implementation.
\begin{figure}[htb]
 \begin{center}
  \includegraphics[width=3in]{UML_Manager.png}
  \caption{Manager UML}
  \label{Manager UML}
 \end{center}
\end{figure}
The \texttt{Manager} module manages instances of \texttt{Encryption}, \texttt{Network}, and
\texttt{User Interface Module}. A user request operations, which the \texttt{Manager} will
process and perform transactions between the other modules. The \texttt{Manager} does not directly
operate on the specific operation, rather, utilizes public operators provided by the support
modules to perform the required transaction.

\subsection{User Interface Module}
\subsubsection{User Interface Class}
The User Interface class provides the public operators for the \texttt{Manager} module to create
and manage multiple instances of the \texttt{User Interface}.

\subsection{Encryption Module}
\subsubsection{Encryption Class}
Figure~\ref{Encryption UML} outlines the Encryption class implementation.
\begin{figure}[htb]
 \begin{center}
  \includegraphics[width=3in]{UML_Encryption.png}
  \caption{Encryption UML}
  \label{Encryption UML}
 \end{center}
\end{figure}
The \texttt{Encryption} module provides support to the \texttt{Manager} for encrypting and
decrypting messages, and for RSA and AES key exchange. The \texttt{Manager} will request key
establishment and will be provided with keys for encryption and decryption.

\subsection{Network Module}
\subsubsection{Network Class}
Figure~\ref{Network UML} outlines the Network class implementation.
\begin{figure}[htb]
 \begin{center}
  \includegraphics[width=3in]{UML_Network.png}
  \caption{Network UML}
  \label{Network UML}
 \end{center}
\end{figure}
The \texttt{Network} module provides support to the \texttt{Manager} for sending and receiving
network traffic, and connection establishment. Connection establishment is done by through a
handshake. Clients must connect to another actively-listening client and once connected will start
handshake messages that will indicate successful network connection. The client can now send or
receive messages. All message packets have been prefixed with appropriate information, indicating
the packet type. \textit{Message}-type packets will be directly passed to the Manager. All other
packets will be processed with the appropriate action conducted by the \texttt{Network}.

% \section{Analysis}

% \section{Results and Discussions}

\appendix
\section{Network} % Appendix A
This appendix considers the \texttt{Network} module in detail.

\subsection{Network Packet Overview}
The \texttt{Network} module utilizes a custom packet format sent across the network implemented
using TCP~\cite{Postel:rfc793}, shown in Figure~\ref{Network Packet Overview}. The figure outlines
the general packet format, which dictates what is provided to the \texttt{Manager}. Figure
~\ref{Network Key Packet} shows how key generation is sent. Key generation requires Establishing
keys on both clients, which is done via sending a \texttt{KEYSIZE} and \texttt{KEY} packet.
\begin{figure} [htb]
 \centering
 \includegraphics[width=3in]{NetworkPacketOverview.png}
 \caption{Network Data Packet}
 \label{Network Packet Overview}
\end{figure}

\begin{figure} [htb]
 \centering
 \includegraphics[width=3in]{NetworkKeyPacket.png}
 \caption{Network Key Packet}
 \label{Network Key Packet}
\end{figure}

\begin{itemize}
 \item \textbf{Network Headers} is used by the \texttt{Network} module. These headers are not
       generally accessible externally, and are used only to process incoming, and tag
       outbound packets. Headers are a single byte long and are guaranteed to be used on all
       network traffic, originating from the \texttt{Network} module.
 \item \textbf{Optional Headers} as of Phase 3, \texttt{DEPRECATED}.
 \item \textbf{Size} As of Phase 3, this field was used to tell the \texttt{Network} how many
       bytes to read from the current message
 \item \textbf{Data} is the data stream to send across the network. \texttt{Network} does not
       access or alter the information.
\end{itemize}

\subsection{Network Header Summary}
The \texttt{Network} module uses the following header options, described in Table~\ref{Network
Header Summary}.
\begin{table}[htb]
 \centering
 \caption{Network Header Summary}
 \label{Network Header Summary}

 \begin{tabular}{|p{4em}|p{20em}|} \hline
  \textbf{Header}  & \textbf{Description}                                  \\ \hline
  \texttt{NOOP}    & used as a no-operation, default network case on error \\ \hline
  \texttt{HELLO}   & used for initial handshake                            \\ \hline
  \texttt{MSG}     & used to indicate transmission of data                 \\ \hline
  \texttt{QUERY}   & \texttt{DEPRECATED}                                   \\ \hline
  \texttt{QUIT}    & used to indicate quitting                             \\ \hline
  \texttt{KEY}     & used to send key                                      \\ \hline
  \texttt{KEYSIZE} & used to send keysizes                                 \\ \hline
  \texttt{ACK}     & \texttt{DEPRECATED}                                   \\ \hline
  \texttt{NACK}    & \texttt{DEPRECATED}                                   \\ \hline
 \end{tabular}
\end{table}
\newline

\subsection{Network Header Detailed Description}
Table~\ref{Network Header Details} lists detailed information on the Network Header options
available.
\begin{table}[htb]
 \centering
 \caption{Network Header Details}
 \label{Network Header Details}

 \begin{tabular}{|c|c|} \hline
  \textbf{Network Header} & \textbf{Data}       \\ \hline
  \texttt{NOOP}           & \texttt{IGNORED}    \\ \hline
  \texttt{HELLO}          & \texttt{IGNORED}    \\ \hline
  \texttt{MSG}            & \texttt{MESSAGE}    \\ \hline
  \texttt{QUERY}          & \texttt{DEPRECATED} \\ \hline
  \texttt{QUIT}           & \texttt{IGNORED}    \\ \hline
  \texttt{KEY}            & \texttt{KEY}        \\ \hline
  \texttt{KEYSIZE}        & \texttt{key size}   \\ hline
  \texttt{ACK}            & \texttt{DEPRECATED} \\ \hline
  \texttt{NACK}           & \texttt{DEPRECATED} \\ \hline
 \end{tabular}
\end{table}

\section{Encryption}
This appendix considers the \texttt{Encryption} module in detail.
\subsection{Key Generation and Exchange}
When a connection is established between two clients, both clients create their public and private
keys. Once the connection is established, the connecting clients starts the key exchange process.
The key exchange begins when the connecting client sends their public key to the other client. The
other client responds with their own key. This public key is used to encrypt the AES key for the
specific client connected. Once completed, the clients are now able to exchange information.
\subsection{Encrypting and Decrypting Messages}
Once keys are established by both clients, messages can be encrypted and decrypted by the AES
public key. In consideration of the whole \texttt{Network} packet, data from the \texttt{DATA
FIELD} are encrypted. It is possible to read other packet metadata.

\balance
\bibliographystyle{abbrv}
\bibliography{report}

\end{document}
